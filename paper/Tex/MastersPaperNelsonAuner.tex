\documentclass[12pt]{article}
\topmargin= -0.4in
\textheight = +8.9in
\oddsidemargin = 0.05in
\evensidemargin = 0.05in
\textwidth = 6.5in
\usepackage{times}
%\usepackage{babel}
\usepackage[capposition=top]{floatrow}
\usepackage{graphicx}
\usepackage[T1]{fontenc}
\usepackage[flushleft]{threeparttable}
\usepackage{lmodern}
\usepackage{graphicx}
\usepackage{amssymb,amsmath}
\usepackage{ifxetex,ifluatex}
\usepackage{fixltx2e} % provides \textsubscript
% use upquote if available, for straight quotes in verbatim environments
\IfFileExists{upquote.sty}{\usepackage{upquote}}{}
\ifnum 0\ifxetex 1\fi\ifluatex 1\fi=0 % if pdftex
  \usepackage[utf8]{inputenc}
\else % if luatex or xelatex
  \ifxetex
    \usepackage{mathspec}
    \usepackage{xltxtra,xunicode}
  \else
    \usepackage{fontspec}
  \fi
  \defaultfontfeatures{Mapping=tex-text,Scale=MatchLowercase}
  \newcommand{\euro}{€}
\fi
% use microtype if available
\IfFileExists{microtype.sty}{\usepackage{microtype}}{}
\usepackage{longtable,booktabs}
\ifxetex
  \usepackage[setpagesize=false, % page size defined by xetex
              unicode=false, % unicode breaks when used with xetex
              xetex]{hyperref}
\else
  \usepackage[unicode=true]{hyperref}
\fi
\hypersetup{breaklinks=true,
            bookmarks=true,
            pdfauthor={},
            pdftitle={},
            colorlinks=true,
            citecolor=blue,
            urlcolor=blue,
            linkcolor=black,
            pdfborder={0 0 0}}
\urlstyle{same}  % don't use monospace font for urls
\setlength{\parindent}{0pt}
\setlength{\parskip}{6pt plus 2pt minus 1pt}
\setlength{\emergencystretch}{3em}  % prevent overfull lines
\setcounter{secnumdepth}{0}

% Alter some LaTeX defaults for better treatment of figures:
    % See p.105 of "TeX Unbound" for suggested values.
    % See pp. 199-200 of Lamport's "LaTeX" book for details.
    %   General parameters, for ALL pages:
    \renewcommand{\topfraction}{0.9}	% max fraction of floats at top
    \renewcommand{\bottomfraction}{0.8}	% max fraction of floats at bottom
    %   Parameters for TEXT pages (not float pages):
    \setcounter{topnumber}{2}
    \setcounter{bottomnumber}{2}
    \setcounter{totalnumber}{4}     % 2 may work better
    \setcounter{dbltopnumber}{2}    % for 2-column pages
    \renewcommand{\dbltopfraction}{0.9}	% fit big float above 2-col. text
    \renewcommand{\textfraction}{0.07}	% allow minimal text w. figs
    %   Parameters for FLOAT pages (not text pages):
    \renewcommand{\floatpagefraction}{0.7}	% require fuller float pages
	% N.B.: floatpagefraction MUST be less than topfraction !!
    \renewcommand{\dblfloatpagefraction}{0.7}	% require fuller float pages

	% remember to use [htp] or [htpb] for placement










\begin{document}

\noindent
\thispagestyle{empty}
\underline{\bf Master's Paper of the Department of Statistics, the
  University of Chicago} 


\vspace{1.8in}
\begin{center}
{\bf\LARGE Combining latent
topics with document attributes in text analysis} \\~\\
%{A parallelizeable, computationally-minded approach } %Let's stew on this one...
%\\~\\
%{\bf\Large --- A Sample Format}


\vspace{1.4in}
{\Large Nelson Auner}

\vspace{1.3in}
{\Large Advisors: Prof. Matt Taddy, Prof. Stephen Stigler \\{\small }}

\end{center}

\vspace{.6in}
{\Large Approved} ~\underline{~~~~~~~~~~~~~~~~~~~~~~~~~~~~~~~~~~~~~~~~~~~~~~
~~~~~~~~~~~~~~~~~~~~~~~~~~~~~~~~~}

\vspace{.2in}
{\Large Date} ~~~~~~~~~\underline{~~~~~~~~~~~~~~~~~~~~~~~~~~~~~~~~~~~~~~~~~~~~~
~~~~~~~~~~~~~~~~~~~~~~~~~~~~~~~~~~~}


\newpage
\pagestyle{plain}
\setcounter{page}{1}

\begin{abstract}

\vspace{7mm}\noindent 

This paper introduces a variant to existing models of multinomial
regression for text analysis. Using the base model introduced by Taddy
(2013), we extend the data-generating model to incorporate topics not
explained by metadata. In doing so, we seek to increase
the prediction accuracy over existing techniques, bridge the gap between
multinomial regression and standard topic models, and investigate
methods for discovering new topics in a corpus. We explore computational
aspects of our approach, provide software for parallelization of the
algorithm, and conclude by proposing areas of future research.

\end{abstract}

%\newpage
\vspace{1.5in}
\tableofcontents


\newpage


\section{Introduction}

\subsection{Text Data}



%\includegraphics{chick}

A common technique for modeling text data is the use of multinomial models on word stems derived from the original document.
Typically, text information is naturally grouped by documents, and each
document is represented by counts of words, or "tokens". A document might be a
single written text (e.g.~an academic article), or a collection of works
by the same author (e.g.~all of the lyrics of an album by the rolling
stones) A token is often a single word (called unigram) but may also be
a sequence of two or more words (e.g.~bi-grams, like `good swimmer' is a
bigram, or tri-grams, like `I eat cheese'). The word components of
tokens are often reduced to a root form by removing suffixes (e.g.
`illuminated', `illumination' and `illuminating' all become
`illuminate').

These tokens are then aggregated by document: For $i$ in $i = 1,...,N$,
the count vector $x_i = [x_{i1}, x_{i2}, ... , x_{ip}]$ contains the
number of occurrences of first, second, \ldots{} $p$ th token in the
$i$th document, where $p$ is the total number of unique tokens in all
documents. This forms the complete count matrix $X$, where each $x_{ij}$
is the number of occurences of word $j$ in document $i$.

Since the number of unique words that appear in a large number of
documents can be extensive, we often restrict the number of tracked
tokens, $p$ to words that occur in at least two documents. We may also
remove common tokens that add little meaning and are found in all
documents (i.e. `the' or `of').

A trivial example of such content might be student's answers to the
question ``What did homework assignments involve?'', with the following four responses:

\begin{table}[!hbpt]
\caption{Example of text data from course reviews} \label{tab:title}
\begin{center}
\begin{tabular} {c c}
\textbf{Document} & \textbf{Content} \\
\hline
1 & Some computation and formula proving, a lot of R code \\
2 & Problems, computation using R \\
3 & Some computations and writing R code\\
4 & Proofs, problems, and programming work \\
\end{tabular}
\end{center}
\end{table}



After removing common words and stemming the remaining words, we might
produce the count matrix in Table 2. 


\begin{table}[!hbpt]
\caption{Creating a word-count matrix from text}
\begin{center}
\resizebox{\textwidth}{!}{%
\begin{tabular}{ c |  c c c c c c c c c c c}
\hline
\textbf{Document} & Some & comp & formula & prov & R & code & use &
problem & writ & program & work \\
1 & 1 & 1 & 1 & 1 & 1 & 1 & 0 & 0 & 0 & 0 & 0 \\
2 & 0 & 1 & 0 & 0 & 1 & 0 & 1 & 1 & 0 & 0 & 0 \\
3 & 1 & 1 & 0 & 0 & 1 & 0 & 0 & 0 & 1 & 0 & 0\\
4 & 0 & 0 & 0 & 1 & 0 & 0 & 0 & 1 & 0 & 1 & 1\\
\hline
\end{tabular}}
\end{center}
\end{table} 



\subsection{Multinomial Model}\label{multinomial-model}

We then model each document $x_i$ as the realization of a multinomial
distribution. That is,

\[ x_{i} \sim MN(q_i,m_i) \]

Where $q_i$ is the vector $[q_{i1}, \dots q_{ip}]$ of token
probabilities for document $x_i$ and $m_i$ is
$\sum_{j = 1}^{p}{x_{ij}}$, or the total number of tokens in document
$i$

It is trivial to show that the maximum likelihood estimator of $q_i$ is
$f_i$ = $x_i / m_i$, but by imposing structure on $q_i$, we can model
features of the data. The two most common techniques for creating
structure are \emph{topic models} and \emph{metadata}

\subsection{Topic Models}\label{topic-models}

A topic model structure assumes that each document is created from a
linear combination of $K$ topics. Each topic $l = 1,..,K$ represents a
distribution, or vector of probability weights
$\omega_l = [\omega_{l1}, ... , \omega_{lp}]$, over words. As a simple
example, we can imagine a fitness store that primarily sells books on
biking, running, and swimming. We can see that a probability
distribution of these topics would have high probability weights on the
terms (``pedal'', ``helment'') for biking, (``stride'') for running, and
(``breath'', ``stroke'', ``water'') for swimming. By denoting the
proportion of topic $l$ as $\theta_l$, we can imagine each document as
being generated by a linear combination of topics
$\omega_1 \theta_1 + \omega_2 \theta_2 + \omega_3 \theta_3$, described
as the following data-generating process:

\begin{enumerate}
\def\labelenumi{\arabic{enumi}.}
\itemsep1pt\parskip0pt\parsep0pt
\item
  Choose $\theta = \theta_1,...\theta_K$ the proportion of topics.
  (i.e., a book completely about swimming would have $\theta=(1,0,0)$ ,
  a book about triathalons might have $\theta =(1/3,1/3,1/3)$ ).
\item
  Choose $m_i$, the number of words in the document
\item
  For each word $j \in 1,2,....,m_i$, choose topic $l$ with probability
  $\theta_l$. With the corresponding weighting vector $\omega_l$, choose
  a word $x_{ij}$
\end{enumerate}

Traditionally, the topic model proportions are given a dirchelet prior, and the model is also known as Latend Dirichlet Allocation, or LDA. 
For a thorough introduction to topic models,we refer the reader to Blei, Ng, and Jordan (2003). 

\subsection{Metadata}\label{metadata}

Text data is frequently accompanied by information, or metadata, about
the text itself. For example, in academic journals, metadata on an
article could include the number of times the article has been cited,
and the journal in which the article has been published. When this
metada is believed to be relevant to the composition of the document, we
use the generic term \emph{sentiment}. For example, given a database of
written movie reviews and final rating out of five $y \in (1,2,3,4,5)$,
we might want to model the relationship between the words used in the
document and the final rating.

\subsection{Metadata and Unigram Models}

If the support $Y$ of metadata $y$ takes discrete values $y^{(1)}, y^{(2)}, \dots, y^{(m)}$ with few unique observations ($m$ small), large computational gains can be had by collapsing the token counts over levels of metadata.  That is, each dataset of $n$ ordered (text, metadata) pairs

\begin{equation}
\big[ (x_1,y_1), (x_2,y_2), \dots , (x_n,y_n) \big]
\end{equation}

can be expressed as $m$ collapsed observations:

\begin{equation}
\big[  (\sum_{x_i : y_i = y^{(1)}}{x_i}) , (\sum_{x_i : y_i = y^{(2)}}{x_i}), \dots,  (\sum_{x_i : y_i = y^{(m)}}{x_i})  \big]
\end{equation}

Then, a simple log-link model allows us to express text $x$, with given metadata rating $y$,
denoted $x_y$, as $x_y \sim MN(q_y,m_y)$, with
\begin{equation}
q_{yj} \sim \frac{exp[\alpha_j + y \phi_j]}{\sum_{l=1}^{p} exp[\alpha_l + y \phi_l]}
\end{equation}

\section{Theory and Approach}\label{theory-and-approach}

\subsection{Mixture models and cluster
membership}\label{mixture-models-and-cluster-membership}

We now turn our attention to the main purpose of this paper, which is to
incorporate latent topics across documents while maintaining the computational simplicity of a collapseable multinomial model. To do so we will restrict our model by assuming that every document is a member of one and only one topic. In order not to confused our approach with a traditional topic models, where each topic can take a weight $\theta \in (0,1)$, we refer to the model in which a document can only belong to one topic as a "cluster membership". This also emphasizes the theoretical relationship between our model and finite mixture models. 

As a simple motivating example, we might imagine a corpus of
movie reviews written by several bloggers. After accounting for text
information explained by the rating (e.g. ~relating a 5-star rating to
`good plot'), the remaining heterogeneity in the movie review
content could be related traits of the blogger (e.g.~gender, or home
city) We may be interested in using predicting traits about bloggers
given their movie reviews, and also in determining how movie review
content changes across these traits.




\subsection{Model Specification}\label{model-specification}

Denoting the word count of a document as the vector $x_i$, we propose
that words in a document are distributed as a multinomial with a
log-link to related sentiment and cluster membership. That is:

\begin{equation}
 x_{i} \sim MN(q_{ij},m_{ij})    ; ~~  q_{ij} = \frac{exp(\alpha_j + y_i \phi_j + u_i \Gamma_{kj})}{\sum_{l=1}^{p}{exp(\alpha_l+ y_i \phi_l + u_i \Gamma_{kl})}}
\end{equation} 


where $y_i$, $u_i$ are the metadata and cluster membership associated
with document $i$, and $\phi_j$ and $\Gamma_{kj}$ are the distortion
coeffecients for metadata and cluster membership, respectively. We use
the subscript $k$ to denote that each document $x_i$ is considered a
member of $k = 1,..,K$ clusters, with their own distortion vectors
$\Gamma_1,..,\Gamma_K$

Now, the unigram collapsing can be expressed as $x_{yk}$, with

\begin{equation}
 x_{yk} \sim MN(q_{yk},m_{yk}), q_{yk} = \sum\limits_{i: y_i = y, u_i = k} \big[\sum_{j = 1}^{p} {x_{ij}} \big]
\end{equation}
and 
\begin{equation}
q_{yk} \sim \frac{exp[\alpha + y \phi  + u_k \Gamma_{k}]}{\sum_{l=1}^{p} exp[\alpha_j + y \phi_j+ u_k \Gamma_{kj} ]}
\end{equation}

\subsection{Initializing Cluster Membership}

Our focus in on predicting cluster membership $u_i$ and the
corresponding probability distortion $\Gamma_i$. Because each document can only be a member of one topic (unlike a traditional topic model), we want to investigate how important the cluster member initialization is to the final coeffecients. 
Forthis paper, we initiatilize cluster membership using one of the three following
methods:

\begin{enumerate}
\def\labelenumi{\arabic{enumi}.}
\itemsep1pt\parskip0pt\parsep0pt
\item
  Random Initialization
\item
  K-means on the word count data $X$
\item
  K-means on the residual of the word count data after incorporating
  metadata $y$ (That is, given predicted word count $\hat{X}$,
  clustering on $X-\hat{X}$
\end{enumerate}

\subsection{Estimation of Parameters via Maximum a
Posteriori}\label{estimation-of-parameters-via-maximum-a-posteriori}

The negative log likelihood of a multinomial distribution can be written
as

\begin{equation} 
\ell(\alpha,\phi,\Gamma,u) = \sum_{i = 1}^{N}{ x_i^\top (\alpha + \phi v_i + u_i \Gamma_{kj})} - m_i log(\sum_{j = 1}^{p}{exp{\big[ \alpha + \phi v_i + u_i \Gamma_{kj} \big]}})
\end{equation}

Following previous literature on text regression (Taddy, 2013), We specify laplace priors and gamma hyperprior on coeffecients, as well as a gamma lasso penalty on coeffecients $c(\Phi,\Gamma)$. This
procedure leads us to minimize

\begin{equation}
\ell(\alpha,\Phi,\Gamma,u) + \sum_{j=1}^{p}(\alpha_j/ \sigma_\alpha)^2 + c(\Phi,\Gamma) 
\end{equation}
The basic algorithm we use to fit coefficients $\alpha$, $\phi$, $\Gamma$
and cluster memberships $u_i$ is two main steps iterated until
convergance:

\textbf{Algorithm for Cluster Membership Model with Gamma Lasso Penalty}
\begin{enumerate}
\def\labelenumi{\arabic{enumi}.}
\item
 Initialize $u_i$ for $i = 1, \dots, n$
\item
  Determine parameters $\alpha_, \phi, \Gamma$ by fitting a multinomial
  regression on $y_i | x_i , u_i$ with a gamma lasso penalty
\item
  For each document $i$, determine new cluster $u_i$ membership as \\
  $argmax_{k = 1,..,K} \big[  \ell(u_i| \alpha, \phi, \Gamma) \big]$
\item 
Check if current cluster assignment is different from previous cluster assignment , ($\textbf{u}^{(t)}  = \textbf{u}^{(t-1)}$).If so, return to step 2. If not, end algorithm.
\end{enumerate}


By alternating between the first two steps, we aim to converge to optimal
parameter estimates $\alpha, \phi, \Gamma$ as well as optimal cluster
membership $u$.
\subsection{Computation}\label{computation}
As noted previously, multinormal regression enjoyes the ability of being able to
collapse observations across levels of metadata. This attractive
property is preserved in the cluster membership model.

In step two, we can increase the speed of step two by only evaluating portions of the likelihood function relevant to $u_i$ and $\Gamma$ by eliminating first two terms from equation one:

\begin{equation} 
L(u_i|\alpha,\phi,\Gamma) = \sum_{i = 1}^{N}{ x_i^\top (u_i \Gamma_{kj})} - m_i log(\sum_{j = 1}^{p}{exp{\big[ \alpha + \phi v_i + u_i \Gamma_{kj} \big]}})
\end{equation}

In addition, the right hand side does not depend on $x_i$ and can be precalculated for each cluster $u_i$. This will lead to an order-of-magnitude speed-up as long as the number of clusters is relatively small compared to the number of documents. 



\section{Application and Evaluation of Algorithm}\label{application}

We applied the Cluster Membership model to two datasets; Congressional Speech records, most famously used to investigate media slant (Moskowitz and Shapiro, 2010) and a corpus of restaurant reviews called we8there.

\subsection{Congressional Speech} 

We first investigate the performance of the algorithm on the congressional speech data of. The data consists of text from 579 speeches of the members of the 109th Congress. For the analysis, party membership was regressed onto speech data. The algorithm was run for 5, 10, 15, 20, and 25 clusters, each over the 3 different initialization methods. We then report the multinomial deviance, or two times the negative log likelihood, in figure 1. 

\begin{figure}[!htpb]
  \centering
  \includegraphics[width=6.2in]{Images/mdev_both.pdf}
  \caption[Multinomial Deviance: Congress Data]
   {Multinomial Deviance from fitted model cluster membership model}
\floatfoot{1 : random cluster initialization, 2 : K Means, 3 : K means on residuals}
\end{figure}


We note that, for any given number of clusters, a better-fitting model is almost always obtained by initializing the cluster memberships on the text data, compared to assigning each document a random cluster.
We also note that initializing the cluster memberships to the residuals of the text data regressed on the outcome variable (in this case, GOP party membership) usually produces a worse-fitting model than simply initializing membership on the original text data.


Previous research (Taddy, 2012) has shown that the optimal number of topics for a topic model on this dataset is around 10. This fact is not shown in our data for a couple of reasons. First, we are possibly overfitting the data, since these models are not run under cross-validation. Second, due to its unigram design, the cluster membership model is much less flexible than a standard topic model, and would likely need many more clusters to model the same lexical variation.

%Do this later dog. 1
\subsection{Evaluation of Cluster Initialization} 
textbf{Possibly remove}
%Method 2: \\
%Clustering the congressional members by speech content led to the overwhelming (majority around 450) being placed in the same cluster, with 1, 3, and 74 persons in the other clusters. After running the algorithm, clusters seemed to be more or less random
%\\
%Method 3:\\
%Clustering the congressional members by unexplained speech content (that is, on the residuls of the regression of party affiliation led to similar results, albiet that the size of the main cluster increased from 450 to 500 congressional members. 
%After the algorithm, 

\subsection{Convergence and Stability of Clusters}

textbf{Possibly remove}
Of key interest is knowing whether or not the algorithm converges to the same clusters when run repeatedly on the same data, and also how this convergence is affected by the 3 proposed cluster initialization.

\subsection{Interpretation of Topics}

An essential aspect of topic modeling is determining the overall theme from the loadings vector. Previous research (Blei and Lafferty, 2009) shows that correctly-specified topic models allow for rich interpration of themes that can change over time, as well as model relationships between themes. 

Of particular importance in our model is that our extreme simplication of topic weights does not result in "meaningless" topics. 
If the topics produced by our method bear no relation to the topics produced by more complex topic modeling approaches, then there is little benefit of our model, regardless of computational improvements and model simplicity. 

To test topic fit, we compare topics produced from our method with the topics produced by fitting a topic-only model, via MAP, with 12 topics on Gentzkow and Shapiro's Congressional Data. Fortuneately, we find that topics from our model are, in many cases, similar to topics obtained by traditional methods. For example, the following table shows a "stem cell" topic found by our method, compared to a similar topic found using the topic-only model mentioned above.

\begin{table}[!htbp]
\begin{threeparttable}
\caption{Comparison of top word loadings on a stem-cell topic} \label{tab:title}
\centering
\begin{tabular}{  c  c }
Cluster Membership & Topic Model (LDA)* \\
\hline
umbilic.cord.blood & pluripotent.stem.cel \\
cord.blood.stem  & national.ad.campaign \\
blood.stem.cel   & cel.stem.cel \\
adult.stem.cel & stem.cel.line \\
\end{tabular}
\begin{tablenotes}
\small
\item  *Results reported in Taddy (2012)
\end{tablenotes}
\end{threeparttable}
\end{table}

\subsection{Interpretation of results}

We first evaluate our model by comparing the coeffecients predicting GOP to a gamma lasso regression without any topic models. 
The words with the highest loading for determing party affiiliation are illustrated in Table 4. 


% latex table generated in R 3.0.2 by xtable 1.7-1 package
% Sat May 03 15:01:23 2014
\begin{table}[!htbp]
\begin{threeparttable}
\caption{Words with highest loadings for predicting Republican-party affiliation}
\centering
\begin{tabular}{r l l | l  l }
 & \multicolumn{2}{c}{Cluster Membership} &  \multicolumn{2}{c}{Multinomial Regression}  \\
  \hline
 & term & loading & term & loading \\ 
  \hline
1 & ready.mixed.concrete & 9.25 & un.official & 5.47 \\ 
  2 & driver.education & 7.34 & people.middle.east & 5.47 \\ 
  3 & speaker.table & 7.2 & speaker.table & 5.47 \\ 
  4 & medic.liability.reform & 6.85 & term.care.insurance & 5.47 \\ 
  5 & near.retirement.age & 6.42 & weapon.grade.plutonium & 5.46 \\ 
  6 & weapon.grade.plutonium & 6.23 & national.homeownership.month & 5.46 \\ 
  7 & death.tax.repeal & 5.98 & nation.oil.food & 5.45 \\ 
  8 & commonly.prescribed.drug & 5.72 & united.nation.oil & 5.45 \\ 
  9 & national.ad.campaign & 5.69 & national.heritage.corridor & 5.44 \\ 
  10 & national.homeownership.month & 5.37 & feder.air.marshal & 5.42 \\ 
\end{tabular}
\begin{tablenotes}
\small
\item *Mixed model fit with 15 topics, each topic initialized with K-means on the word count matrix
\end{tablenotes}
\end{threeparttable}
\end{table}


The theory behind our mixed topic model regression predicts that the topics should be able to incorporate a specific theme important to a group of individuals, leaving behind the more general predictors of party-affiliation to the regression coeffecient.
We can test this prediction by examining terms that are significant for a Multinomial Regression without topics, but decrease in importance once topic models are added. For a simple illustration, we choose the phrase "nation oil food", which predicts affiliation with the republican party.  The term is strongly associated with a topic we might call "domestic issues" topics, with the following high word loading terms:

%We find that these terms are now associated with specific topics and specific groups of politicians, instead of being confounded with general party-affiliation.




% latex table generated in R 3.0.2 by xtable 1.7-1 package
% Wed May 07 18:31:04 2014
\begin{table}[ht]
\centering
\begin{threeparttable}
\begin{tabular}{rll}
  \hline
 & term & loading \\ 
  \hline
1 & nation.oil.food & 20.09 \\ 
  2 & united.nation.oil & 12.09 \\ 
  3 & liberty.pursuit.happiness & 8.11 \\ 
  4 & life.liberty.pursuit & 8.11 \\ 
  5 & minority.women.owned & 6.73 \\ 
  6 & universal.health & 6.67 \\ 
  7 & white.care.act & 6.64 \\ 
  8 & ryan.white.care & 6.6 \\ 
  9 & universal.health.care & 5.99 \\ 
  10 & growth.job.creation & 5.39 \\ 
  11 & drilling.arctic.national & 5.3 \\ 
  12 & tax.relief.package & 5.29 \\ 
  13 & judge.john.robert & 5.26 \\ 
  14 & fre.enterprise & 5.07 \\ 
  15 & arctic.refuge & 4.93 \\ 
   \hline
\end{tabular}
\begin{tablenotes}
\small
\item One cluster from a model fit with 20-clusters , each having been initialized with K-means on the residuals of metadata regression
\end{tablenotes}
\end{threeparttable}
\end{table}


We also notice that the members of this cluster (by our simplification, each observation can only be a member of one topic) are 3 republicans and 8 democrats. 
The ability to group observations that may have different metadata (in this case, political party) is a benefit of our mixed regression-topic model approach. 
Under standard multinomial regression, observations cannot be grouped by topics, whereas topic modeling does not immediately offer a computationally simple way to include influence on outside metadata.

\subsection{Comparison to Blei's Inverse Regression Topic Model}
As mentioned in Blei (IRTM) the addition of latent topics to a model of text data with attributes is the ability to gain an intuitive concept to how metadata affects the distribution of a given topic. 
To demonstrate this effect, we use the graphic model presented in (Blei) to show the Democrat/GOP distortion to the "domestic issues"  topic shown earlier. 

\begin{figure}[!htpb]
  \centering
\caption[Loadings]{Cluster word loadings with covariate distortion}
  \includegraphics[width=6.2in]{Images/Blei_Changing_Loadings_GOP.pdf}
\floatfoot{A cluster from the Congressional data. On the left are, from highest to lowest, terms with the top Democratic word loadings. On the right are the words with the highest Republican word loadings. The horizontal position indicates the value of $\phi$, or word distortion for Republicans}
\end{figure}


\subsection{We8there Data}

We briefly illustrate the results of the cluster membership model on the we8there corpus of restaurant reviews with a table of word loadings for a selected cluster, as well as a distortion graph of that topic. 


\section{Extension}\label{extensions}

\subsection{Feature Allocations}
A promising extension of the cluster membership model is the generalized "feature allocation" (Broderick 2014), where each observation can be attributed multiple features. This setup is an intermediary between our cluster membership model and a traditional topic model. 
The algorithm provided in this paper could be extended to incorporate a feature allocation, although the order of step 3 of the algorithm increases from linear $\mathcal{O}_{cluster} = n_k$ to $\mathcal{O}_{feature} = 2^{n_k}$, where $n_k$ refers to the number of clusters/features in the model. This increase in complexity may be reduced through changes to the algorithm, and the model would still enjoy the ability to collapse observations across unique features. 

\subsection{Cross Validation and Prediction}
One drawback of our is that, in order to train cluster membership, the values of the metadata are required (see algorithm pseudocode in previous section). This hinder the ability to use the method for prediction (where, presumably, we hope to predict missing metadata from the text), and impeded our ability to perform cross validation. One possible solution for this is, after fitting a model with training data, use k-nearest neighbors or other appropriate algorithm to assign each document of the test data to a cluster. However, this approach would be difficult to combine with the feature allocation extension proposed above, since the $n$ observations would be seperated into $2^{n_k}$ partitions, instead of $n_k$. 



\section{Conclusion}\label{conclusion}

In this paper, we have reviewed the theory behind topic modeling and regression on metadata in text data. 
We introduce an algorithm that combines the metadata regression techniques developed by Taddy (2013a, 2013b) with a simple adaptation of the classic topic model. We then 


\section{Acknowledgements}\label{conclusion}

I thank Professor Matt Taddy for his guidance throughout the research and writing of this paper. I also thank Professor Stephen Stigler for his feedback, the University of Chicago Research Computing Center (RCC) for computational resources, and Eric Janofsky for the helpful discussion. 


\newpage

%\begin{appendix}
%\section{Appendix}

%\vspace{4mm}\noindent 
%The following may be included in an appendix:

%\begin{itemize}
%\item[] Data 
%\item[] Simulation codes
%\item[] Certain derivations
%\end{itemize}
%\end{appendix}
%\newpage

\begin{thebibliography}{99}

\bibitem{TopicModelsTaddy} Taddy (2012).On Estimation and Selection for Topic Models.  Proceedings of the 15th International Conference on Artificial Intelligence and Statistics

\bibitem{LDA} Blei, David M., Ng, Andrew Y., and Jordan, Michael I (2003). Latent Dirichlet allocation. Journal of Machine Learning Research (JMLR), 3:993–1022, March 2003. ISSN 1532-4435

\bibitem{CorrelatedTopicModels}  Blei and Lafferty (2007) Correlated Topic Models. Proceedings of the 15th International Conference on Artificial Intelligence and Statistics

\bibitem{IRTM} Rabinovich and Blei (2014) Proceedings of the 31st International Conference on Machine Learning, Beijing, China, 2014. JMLR: W\&CP volume 32

\bibitem{Paintboxes}T. Broderick, J. Pitman, and M. I. Jordan (2013). Feature allocations, probability functions, and paintboxes. Bayesian Analysis, to appear. Preprint arXiv:1301.6647, 2013

\bibitem{PartialInverseRegression}  Li, Cook  and Tsai. Partial Inverse Regression. Biometrika (2007), 94, 3, pp. 615–625

\bibitem{GammaLasso} Taddy (2013) The Gamma Lasso. The gamma lasso. arXiv:1308.5623

\bibitem{MNIR} Taddy, M. (2013). Multinomial inverse regression for text analysis. Journal of the American Statistical Association 108.

\end{thebibliography}
\end{document}
